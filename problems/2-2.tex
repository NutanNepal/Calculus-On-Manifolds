\question{
    [2-2] A function $f:\rl^2\to\rl$ is
    \textbf{independent
    of the second variable} if for each $x\in\rl$
    we have $f(x,y_1)=f(x,y_2)$ for all $y_1,y_2
    \in\rl$. Show that $f$ is independent of the
    second variable if and only if there is a
    function $g:\rl\to\rl$ such that $f(x,y)=g(x).$
    What is $f'(a,b)$ in terms of $g'$?
}
\begin{solution}
    Define $g(x)=f(x,0)$. Then for all $y\in\rl$,
    if $f$ is independent of the second variable,
    we have $f(x,y)=f(x,0)=g(x)$.

    Similarly, since $g$ is independent of y, we
    have $g(x)=f(x,0)=f(x,y_1)=f(x,y_2)$.

    Now let $z=(h,k).$ Then, assuming that $f$ is
    differentiable
    at $(a,b)$, we have
    \begin{align*}
        &\lim_{(h,k)\to 0}
        {\frac{\|f(a+h,b+k)-f(a,b)-Df(a,b)(h,k)\|}
        {\|(h,k)\|}}=0\\
        \text{or, }&\lim_{h\to 0}
        {\frac{\|g(a+h)-g(a)-Df(a,b)(h,k)\|}
        {\vert h\vert}}=0\\
        \shortintertext{Since $g:\rl\to\rl$,}
        &\lim_{h\to 0}{\frac{g(a+h)-g(a)}
        {\vert h\vert}}=
        \lim_{h\to 0}{\frac{Df(a,b)(h,k)}
        {\vert h\vert}}\\
        \text{or, }&g'(a)=
        \lim_{h\to 0}{\frac{Df(a,b)(h,k)}
        {\vert h\vert}}
        \shortintertext{Then we see that}
        &Df(a,b)(h,k)=h\cdot g'(a)
    \end{align*}
    satisfies the equation. Hence,
    $f'(a,b)=g'(a)$.
\end{solution}