\question{
    [2-8] Let $f:\rl\to\rl^2$. Prove that
    $f$ is
    differentiable at $a\in\rl$ if and
    only if $f_1$ and $f_2$ are, and that
    in this case
    $$f'(a)=\left(\begin{matrix}
        f_1'(a)\\f_2'(a)
    \end{matrix}\right).$$
}
\begin{solution}
    If $f:\rl\to\rl^2$ is differentiable
    at $a$, 
    then for some linear
    transformation $\lambda:\rl\to\rl^2$,
    $$\lim_{h\to 0}
    {\frac{\|f(a+h)-f(a)-\lambda(h)\|}
    {\vert h\vert}}
    =0$$
    So, for every $\varepsilon>0$, there
    exists a $\delta>0$ such that
    $$0<\vert h\vert <\delta\implies
    \|f(a+h)-f(a)\|<
    \vert h\vert\varepsilon.$$
    But this means that each $f_1$ and
    $f_2$ satisfies
    $\|f_i(a+h)-f_i(a)\|<\vert h\vert
    \varepsilon.$ So each $f_i$ is
    differentiable. The converse follows
    similarly.

    Then,
    $$\frac{\|f(a+h)-f(a)-\lambda(h)\|}
    {\vert h\vert}=\left\lVert\left(
        \begin{matrix}
        \frac{f_1(a+h)-f(a)-\lambda_1(h)}
        {h}\\
        \frac{f_2(a+h)-f(a)-\lambda_2(h)}
        {h}
        \end{matrix}\right)\right\rVert$$
    
    We see that each of the component
    of the right
    hand side must be $0$. We also have
    $f_i'(a)=\lambda_i(h)/\vert h\vert.$
    Hence the
    required expression for $f'(a)$
    follows.
    $$\vert h\vert f'(a)=
    \lambda(h) =
    \left(
    \begin{matrix}
        \lambda_1(h)\\
        \lambda_2(h)
    \end{matrix}\right)
    =
    \left(
    \begin{matrix}
        \vert h\vert f_1'(a)\\
        \vert h\vert f_2'(a)
    \end{matrix}\right).
    $$
\end{solution}