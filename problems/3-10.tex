\question{
    [3-10]

    (a) If C is a set of content 0, show
    that the boundary of C has content 0.

    (b) Give an example of a bounded set
    C of measure 0 such that the boundary
    of C does not have measure 0.
}
\begin{solution}
    \begin{parts}
        \part{
            Let A be the set of content 0.
    Then for every $\varepsilon>0$,
    there is a finite cover
    $\{U_1,\ldots,U_n\}$ of A by
    closed intervals such that
    $\sum_{i=1}^n{v(U_i)}<\varepsilon$.
    Also since $A\subset\cup_{i=1}^n{U_i}$,
    where each $U_i$ are closed, boundary
    of $A$ are also contained in the finite
    union and the same $U_i$ suffices to
    conclude that the boundary of $A$ has
    content 0.
        }

        \part{
        A=$\{x\in[0,1]:x\in\mathbb{Q}\}$
        has measure 0. But the
        boundary of A is all of $[0,1]$
        which doesn't have measure 0.
        }
    \end{parts}
\end{solution}