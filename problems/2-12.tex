\question{
    [2-12] A function $f:\ron\times\rom\to\rl^p$
    is $\mathbf{bilinear}$ if for $x,x_1,
    x_2\in\ron$, $y,y_1,y_2\in\rom$, and
    $a\in\rl$ we have
    \begin{align*}
        f(ax,y)&=af(x,y)=f(x,ay),\\
        f(x_1+x_2,y)&=f(x_1,y)+f(x_2,y),\\
        f(x,y_1+y_2)&=f(x,y_1)+f(x,y_2).
    \end{align*}
}
\begin{parts}
    \part{
        Prove that if $f$ is bilinear, then
        $$\lim_{(h,k)\to 0}{
            \frac{\|f(h,k)\|}{\|(h,k)\|}
        }=0$$
    }
    \begin{solution}
        Let $h=\alpha_1e_1+...+\alpha_ne_n$
        and $k=\beta_1e_1+...+\beta_me_m$.
        Since $f$ is bilinear, we can write
        $f(h,k)$ as
        $$f(h,k)=\sum_{i=1}^n{
            \sum_{j=1}^m{
                \alpha_i\beta_jf(e_i,e_j)
            }
        }.
        $$
        Let $\alpha=\max{
        \{\vert\alpha_i\vert:
        i=1,...,n\}}$ and $\beta=
        \max{\{\vert\beta_i\vert:
        i=1,...,m\}}.$
        Then $\alpha\le\|h\|$ and
        $\beta\le\|k\|$. Also let
        $M=\max{\{\|f(e_i,e_j)\|:i=1,...,n
        \textnormal{ and } j=1,...,m\}}.$
        So we have,
        \begin{align*}
        \lim_{(h,k)\to 0}{
            \frac{\|f(h,k)\|}{\|(h,k)\|}
        }
        &\le\lim_{(h,k)\to 0}{
            \frac{\vert mn\alpha\beta M\vert}
            {\|(h,k)\|}
        }\\
        &\le\lim_{(h,k)\to 0}{
            \frac{mnM\|h\|\|k\|}{\|(h,k)\|}
        }
        \end{align*}
        Now,
        $$
        \|h\|\|k\|\le
        \begin{cases}
            \|h\|^2 & 
            \textnormal{if }\|k\|\le\|h\|,\\
            \|k\|^2 &
            \textnormal{if }\|h\|\le\|k\|.
        \end{cases}
        $$
        Hence $\|h\|\|k\|\le\|h\|^2+\|k\|^2$
        and since, $\|(h,k)\|=\sqrt{
        \|h\|^2+\|k\|^2}$ we have
        $$\lim_{(h,k)\to 0}{
            \frac{\|f(h,k)\|}{\|(h,k)\|}
        }\le
        \lim_{(h,k)\to 0}{
            \frac{mnM(\|h\|^2+\|k\|^2)}
            {\sqrt{\|h\|^2+\|k\|^2}}
        }
        =\lim_{(h,k)\to 0}{mnM
        \sqrt{\|h\|^2+\|k\|^2}}=0.$$
    \end{solution}
    \part{
        \label{2-12,b}
        Prove that $Df(a,b)(x,y)=
        f(a,y)+f(x,b)$.
    }
    \begin{solution}
        Here, for $a,h\in\ron$ and $b,k\in\rom$,
        \begin{align*}
            &\lim_{(h,k)\to 0}{
                \frac{\|f(a+h,b+k)
                -f(a,b)-\lambda(h,k)\|}{
                \|(h,k)\|}
            }\\
            \le &\lim_{(h,k)\to 0}{
                \frac{\|f(h,k)\|}{\|(h,k)\|}
            }+\lim_{(h,k)\to 0}{
                \frac{\|f(a,b)+f(a,k)+f(h,b)
                -f(a,b)-\lambda(h,k)\|}
                {\|(h,k)\|}
            }
        \end{align*}Then certainly,
        $\lambda(h,k)=f(a,k)+f(h,b)$
        implies,
        $$\lim_{(h,k)\to 0}{
            \frac{\|f(a+h,b+k)
            -f(a,b)-\lambda(h,k)\|}{
            \|(h,k)\|}
        }=0.$$
        Hence, $Df(a,b)(x,y)=f(a,y)+f(x,b).$
    \end{solution}
    \part{
        When $p:\rl^2\to\rl$ is defined by 
        $p(x,y)=x\cdot y$, then
        $$Dp(a,b)(x,y)=bx+ay.$$
        Show that the above formula is a
        special case of \ref{2-12,b}.
    }
    \begin{solution}
        It's clear that $p$ is a bilinear
        function. Hence the derivative of $p$
        at $(a,b)$ is given by,
        \begin{align*} 
            Dp(a,b)(x,y)&=p(a,y)+p(x,b)\\
            &=ay+bx.
        \end{align*}
    \end{solution}
\end{parts}