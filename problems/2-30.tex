\question{
    [2-30]
    Let $f$ be defined as in (\ref{2-4}).
    Show that $D_xf(0,0)$ exists for all
    $x$, but if $g\ne 0$, then $D_{x+y}
    f(0,0)=D_xf(0,0)+D_yf(0,0)$ is not
    true for all $x$ and $y$.

    In the referred exercise:
    $f:\rl^2\to\rl$ is defined by
    $$f(x)=\begin{dcases}
        \lVert x\rVert\cdot g\left(
            \dfrac{x}{\lVert x
            \rVert}\right)
        &x\ne 0,\\
        0& x=0.
    \end{dcases}$$
    where $g$ is a continuous real-valued
    function on a unit circle centered at
    origin in $\rl^2$ such that $g(0,1)=
    g(1,0)=0$ and $g(-x)=-g(x)$.
}

\begin{solution}
    For $x\in \rl^2$,
    $$D_xf(0,0)=
    \lim_{t\to 0}{\dfrac{f(tx)-f(0,0)}
    {t}}=\lim_{t\to 0}{\dfrac{tf(x)}{t}}
    =f(x)$$
    exists. But
    $$D_{x+y}f(0,0)=f(x+y)=\lVert x+y
    \rVert\cdot g\left(\dfrac{x+y}
    {\lVert x+y\rVert}\right)$$,
    and in order for the above equation
    to be equal to \begin{displaymath}
        D_xf(0,0)+D_yf(0,0)=f(x)+f(y)
        =\lVert x\rVert\cdot g\left(
        \dfrac{x}{\lVert x\rVert}\right)
        +\lVert y\rVert\cdot g\left(
        \dfrac{y}{\lVert y\rVert}\right).
    \end{displaymath}
\end{solution}