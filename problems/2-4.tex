\question{
    [2-4] Let $g$ be a continuous real-valued
    function on the
    unit circle ${x\in \mathbf{R}^2:\|x\|=1}$ such
    that $g(0,1)=g(1,0) =0$ and $g(-x)=-g(x)$.
    Define $f:\mathbf{R}^2\to \mathbf{R}$ by
    $$f(x)=
    \begin{cases}
        \|x\|\cdot g\left(\frac{x}{\|x\|}\right)
        & x \neq 0,\\
        0 & x =0.\\        
    \end{cases}$$
}
\begin{parts}
    \part{
        If $x\in \mathbf{R}^2$ and
        $h:\mathbf{R}\to\mathbf{R}
        $ is defined by $h(t)=f(tx)$, show that
        $h$ is differentiable.
    }

    \begin{solution}
        We need to show that for every $a\in
        \mathbf{R}$, there
        exists a $\lambda:\mathbf{R}\to\mathbf{R}$
        such that
        \begin{equation}
            \lim_{t\to 0}
            {\frac{h(a+t)-h(a)-\lambda(t)}{t}} = 0.
        \end{equation}
        We see that, when $tx\neq 0$,
        $$h(t)=f(tx)=
        \begin{cases}
            -\vert t\vert\cdot\|x\|\cdot 
            g\left(\hat{x}\right) = tf(x) & t<0,\\
            \vert t\vert\cdot\|x\|\cdot 
            g\left(\hat{x}\right) = tf(x) & t>0.\\
        \end{cases}$$
        Then $h$ is differentiable when the
        following
        limit exists for any $a\in \rl$:
        $$\lim_{a \to 0 }{\frac{h(t+a)-h(t)}{a}}.$$
        But we have,
        \begin{align*}
            \lim_{a\to 0}{\frac{h(t+a)-h(t)}{a}}
            &=\lim_{a\to 0}{\frac{(t+a)f(x)-tf(x)}
            {a}}\\
            &= f(x).
        \end{align*}
        The limit always exists and is equal to the
        derivative of $h$ at $t$.
    \end{solution}

    \part{
        Show that $f$ is not differentiable at
        $(0,0)$
        unless $g=0$.
    }
    \begin{solution}
        
    \end{solution}
\end{parts}