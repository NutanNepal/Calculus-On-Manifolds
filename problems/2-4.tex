\question{
    [2-4] Let $g$ be a continuous real-valued function
    on the
    unit circle ${x\in \mathbf{R}^2:\|x\|=1}$ such
    that $g(0,1)=g(1,0) =0$ and $g(-x)=-g(x)$.
    Define $f:\mathbf{R}^2\to \mathbf{R}$ by
    $$f(x)=
    \begin{cases}
        \|x\|\cdot g\left(\frac{x}{\|x\|}\right)
        & x \neq 0,\\
        0 & x =0.\\        
    \end{cases}$$
}
\begin{parts}
    \part{
        If $x\in \mathbf{R}^2$ and
        $h:\mathbf{R}\to\mathbf{R}
        $ is defined by $h(t)=f(tx)$, show that $h$ is
        differentiable.
    }

    \begin{solution}
        We need to show that for every $a\in \mathbf{R}$,
        there
        exists a $\lambda:\mathbf{R}\to\mathbf{R}$
        such that
        \begin{equation}
            \lim_{t\to 0}
            {\frac{h(a+t)-h(a)-\lambda(t)}{t}} = 0.
        \end{equation}
        We see that
        $$h(a+t)=f(ax+tx)=
        \begin{cases}
            \|(a+t)x\|\cdot g\left(\frac{x}
            {\|x\|}\right) & (a+t)x\neq 0,\\
            0 & (a+t)x=0.\\
        \end{cases}$$
        This follows from the fact that
        $g\left(\frac{(a+t)x}
        {\|(a+t)x\|}\right)=g\left(\frac{\pm x}{\|x\|}\right)
        =g\left(\frac{x}{\|x\|}\right).$ We have,
        \begin{align*}
            \frac{h(a+t)-h(a)-\lambda(t)}{t} & =
            \frac{\|(a+t)x\|\cdot g(\hat{x})-\|a\|g
            (\hat{x})-\lambda(t)}
            {t}\\
            & = 
        \end{align*}
    \end{solution}
\end{parts}