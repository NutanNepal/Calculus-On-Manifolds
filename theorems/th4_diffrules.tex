\begin{theorem}
    \textnormal{Corollary from the book}\\
    \label{differentiation theorems}
    If $f,g:\ron\to\rl$ are differentiable
    at $a$,
    \begin{align*}
        &D(f+g)(a)=Df(a)+Dg(a)\\
        &D(f\cdot g)(a)
                    =g(a)Df(a)+f(a)Dg(a)\\
        \shortintertext{If, moreover,
        $g(a)\neq 0$, then}
        &D(f/g)(a)=\frac{g(a)Df(a)-f(a)Dg(a)}
        {[g(a)]^2}
    \end{align*}
\end{theorem}

\begin{tcolorbox}
    \begin{proof}
        The first one is done in the text.
        So we'll do the second and the third
        one.\\
        So, using the notations from the text,
        since $f\cdot g =p\circ(f,g)$,
        \begin{align*}
            D(f.g)(a)&=Dp(f(a),g(a))\circ
                D(f,g)(a)\\
            &=Dp(f(a),g(a))(Df(a),Dg(a))\\
            &=g(a)Df(a)+f(a)Dg(a)           
        \end{align*}
        The third relation follows from the
        above product rule
        and \ref{quotientlemma}.
    \end{proof}
    \begin{lemma}
        \label{quotientlemma}
        If $q:\rl\to\rl, g:\ron\to\rl$
        is defined by
        $q(x)=\frac{1}{g}(x)$, then
        $$Dq(a)=-\frac{Dg(a)}{[g(a)]^2}.$$
        \begin{proof}
            We have,
            $q(x)\cdot g(x)=1$. Then,
            $D(1)=q(x)Dg(x)+g(x)Dq(x).$
            Substituting $q(x)=1/g(x)$
            gives the required result.
        \end{proof}
    \end{lemma}
\end{tcolorbox}
